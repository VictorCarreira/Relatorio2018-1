\begin{abstract}

Apresenta-se, neste relatório, o que foi desenvolvido até o presente momento do projeto de doutorado sobre a aplicação da inteligência artificial no reconhecimento de padrões litológicos. Primeiramente, é apresentado a motivação da obra. Posteriormente, é explicado o que vem a ser redes neuronais e ainda apresento trabalhos já publicados e aplicados na área da perfilagem de poços. Em seguia explico os princípios matemáticos envolvidos e apresento a rede escolhida para resolver o problema proposto. Ao final do capítulo $1$, mostro o que vem a ser aprendizado não-supervisionado. O capítulo $2$ esclarece o contexto geológico da área que virá a ser estudada, nas etapas posteriores do projeto, que será a etapa de trabalho com o dado real. No capítulo $3$, é exposto o método que será utilizado ao longo do projeto bem como quais são os seus objetivos. O capítulo $4$ ilustra a natureza do dado de \textit{well logging} e apresenta um teste de hipóteses realizado, na rede neuronal. No capítulo $5$, são mostrados os resultados desse teste, para as etapas de treinamento e identificação da rede. Estes testes apontaram que o erro da rede relativo à etapa do treinamento foi de $4\%$. E a estabilização da rede se deu com $1000$ ciclos de treinamento e com custo computacional de $20$ segundos, na compilação do programa.  Por conseguinte, no capítulo $6$, são apresentadas as conclusões dos testes sintéticos. São publicados, no capítulo $7$,  o cronograma de atividades do projeto atualizado seguido pelas referências bibliográficas.      

\end{abstract}

